\section{Genetic Heterogeneity in Multi-Agent Systems}

\begin{frame}{Defining Heterogeneity}

heterogeneity: NOT absolute homogeneity
\begin{itemize}
\item all agents \textit{completely} identical
\item intuition: standing between two mirrors
\end{itemize}

\vspace{2ex}

necessary to solve leader election problem \cite{angluin1980local,banda2015configuration}

\end{frame}

\begin{frame}{Multiple Sources of Heterogeneity}

heterogeneous groups can solve leader election:
\begin{itemize}
\item configuration \cite{frederickson1987electing}
\item state \cite{banda2015configuration}
\item connectivity \cite{antonoiu1996self}
\item stochasticity \cite{itai1981symmetry}
\end{itemize}

\vspace{2ex}

at least \textit{some} heterogeneity seems essential
\begin{itemize}
\item ... and is ubiquitous in multi-agent systems \cite{atodd2015quantitative, perna2012individual, fayeez2017h}
\end{itemize}


\end{frame}

\begin{frame}{Genetic Heterogeneity}

\begin{itemize}
\item configuration heterogeneity
\item specialization
\begin{itemize}
\item division of labor \cite{potter2001heterogeneity}
\item different capabilities (e.g., ground and aerial robots) \cite{gomes2015cooperative, mathews2012supervised}
\item variation in unit-to-unit hardware \cite{pugh2007parallel, duarte2016evolution}
\end{itemize}
\end{itemize}

alternative: plasticity \cite{tuci2008evolving}

\end{frame}

\begin{frame}{Credit Assignment Problem}

genetics has a specific connotation in evolutionary computing

crux: nailing down value of individual contributions to group performance is nontrivial \cite{panait2005cooperative}

\end{frame}

\begin{frame}{Credit Assignment Problem}

intuition: hockey team

weakens selection for good/against bad solutions
\begin{itemize}
\item intuition: randomly drawn teams of varsity and junior varsity players $\rightarrow$ own performance weakly correlates with team performance
\item problem gets worse as teams get bigger
\end{itemize}

can incentivize defection
\begin{itemize}
\item intuition: if players rewarded for goals they make, they will take as many shots as possible (they make more goals, but team makes fewer goals)
\end{itemize}

requires (appropriate) cooperating partner/context-dependent
\begin{itemize}
\item intuition: player who is great at passing might be on a team without any passers
\end{itemize}

\end{frame}

\begin{frame}{Credit Assignment Problem}

three approaches:
\begin{itemize}
\item designing individual payoffs in order to align with individual contribution to group success \cite{waibel2009genetic}
\item calculating fitness as the difference between group performance and estimated group performance without the contributions of an agent \cite{knudson2010coevolution}
\item cooperative co-evolution, where individuals in distinct subpopulations corresponding to distinct roles are selected by evaluation only with the best-performing individuals from other subpopulations \cite{gomes2015cooperative}
\end{itemize}

\end{frame}
